

\subsection{Mixing Matrix and Its Parametrization}
Neutrinos that we produced and observed are the so-called flavour eigenstates. The name comes from the fact that these states always manifest themselves via the presence of a corresponding charged lepton (the lepton in the same family with the neutrino) in a weak interaction. On the other hand, the mass eigenstates of neutrinos are the ones that possess definite masses. The mixing between the flavour eigenstates and the mass eigenstates can be understood purely by quantum mechanics, using the following mixing matrix:
\begin{equation}
U = R_{32}~R_{31}~R_{21},
\end{equation}
in which,
\begin{equation}
R_{32} = 
\begin{pmatrix}
1 & 0 & 0 \\
0 & \cos\theta_{23} & \sin\theta_{23}  \\
0 & -\sin\theta_{23} & \cos\theta_{23}  \\
\end{pmatrix},
R_{31} = 
\begin{pmatrix}
\cos\theta_{31} & 0 & \sin\theta_{31} \\
0 & 1 & 0  \\
-\sin\theta_{23} & 0 & \cos\theta_{23}  \\
\end{pmatrix},
R_{21} = 
\begin{pmatrix}
\cos\theta_{31} & \sin\theta_{31} & 0 \\
-\sin\theta_{23} & \cos\theta_{23} & 0 \\
\end{pmatrix}.
\end{equation}