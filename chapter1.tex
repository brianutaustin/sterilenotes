\section{Introduction}


%Neutrino, for a long time, has been considered being massless as described in the Standard Model of fundamental particles. The story began when experimental results pointed out that all observed neutrinos were left-handed in \textbf{helicity} (and all anti-neutrinos are right-handed). At the same time, the tiny masses of neutrinos tricked physicists to believe that they are actually massless. In the massless limit, helicities and chiralities coincides. This means all neutrinos are left-handed in \textbf{chirality} (and all the antiparticle counterparts are right-handed in chirality).\footnote{Helicity is a quantity that depends on kinematics. It tells us if the spin of a particle is parallel or antiparallel to its momentum. Chirality, on the other hand, is an intrinsic property of a particle. Roughly speaking, it tells us how the particle (or its quantum field) behaves in a mirror reflection. In the language of theorists, the chirality is determined by whether the particle transforms in a right-handed or left-handed representation of Poincare group. As an example, Dirac spinors have both RH and LH components, the RH component with RH chirality and the LH one with LH chirality.}\\

%Later on, neutrino oscillation is discovered and strongly confirmed. This implies neutrinos have masses, even though they are tiny. The existence of masses, in turn, requires the presence of RH neutrinos, which have never been observed in any experiment. The question, then, becomes whether or not LH and RH neutrinos exist as separate particles?\\


%Questions for neutrino physicists:
%\begin{itemize}
%\item How many neutrino species do we have? Are there any sterile neutrinos? The most prominent indications for sterile neutrinos so far are LSND, MiniBOONE, reactor anomalies.
%\item What is the scale of neutrino masses? Why are neutrinos so light? If the same Higgs mechanism which is responsible for the masses of other fermions is also responsible for the masses of neutrinos, the Yukawa couplings of neutrinos have to be up to 6 orders of magnitude smaller than the one for electron, which produces a non-pleasant hierarchy between the Yukawa couplings.
%\item What is the hierarchy of neutrino masses?
%\item Is there CP-violation in the neutrino sector?  If yes, what is the value of the CP-violating phase?
%\end{itemize}

%LSND experiment:

Since the first time introduced, neutrinos have been always the inspiration behind new discovers in particle physics. As a trade-off, they have also brought about huge challenges for physicists on their way to study their properties. For example, even though theoretically predicted in 1930 by Wolfgang Pauli as the solution for the problem of beta decay, electron neutrinos were only observed in an experiment designed by Clyde Cowan and Frederick Reines, conducted 26 years later. And the existence of the last member in the neutrino family, which is tau neutrino, was just discovered in 2000 in DONUT experiment with merely four signal events. \\

There are many reasons behind these difficulties. One of them is the elusiveness of neutrinos: a neutrino can travel through a light-year of lead without a single trace of interaction. Explanation is that the particles do not take part in neither the strong interaction or the electromagnetic one. They only involve in weak interaction, the weakest force among the three. That is why the particles are so difficult to be detected. \\

Beside the remarkable elusiveness, neutrinos also famous for their light-weights. Actually, they are so light that almost all physicists were once believed that they are massless. We confirmed that the particle possess masses, even though just tiny ones, just recently, via the so-called neutrino oscillation. This phenomenon has not only solved a number of puzzles existed at the heart of physics and astronomy in the twentieth century, but also opened a door to a new physics, suggesting hints to fundamental questions including the cosmological evolution. In this introduction, we will go over the history of neutrino oscillation and try to have a bird-eye view on the current status and the prospects of the field.\\

\subsection{A glimpse at the history of neutrino oscillation}
\subsubsection{Solar neutrino problem}
% Adding why neutrino is used to study the Sun and then later becomes the Solar Neutrino Problem: https://www.nobelprize.org/nobel_prizes/physics/laureates/2015/advanced-physicsprize2015.pdf
The field of neutrino oscillation has a root in astronomy, more precisely from the study of Sun model. All started with the solar neutrino problem. Simply put, the problem results from a smaller observed neutrino flux from the Sun. "Smaller" here is in comparison with the flux predicted by the standard model of the Sun. Due to a strong theoretical and empirical supports on the solar model, this deficit leads to the suspicion in our understanding of neutrinos. \\
% Adding the quantitative description of the problem

The first experiment to measure the neutrino flux from the Sun and detect the deficit is the Homestake Experiment by Ray Davis and John N. Bahcall. The deficit is later confirmed by many subsequent experiments, including the SNO. GALLEX and SAGE, which are radiochemical experiments, also play an important role in solving the solar neutrino problem, but because they, at the same time, shed a light on a much more fascinating aspect of neutrinos, we will spare the discussion of these experiment for a moment.\\

\subsubsection{The SNO experiment}
Historically, the pre-SNO era's detectors were limited by the exclusively sensitivity to electron neutrinos and merely able to confirm the deficit that Homestake has discovered: a third to a half fewer neutrinos than Standard Solar Model's prediction. SNO, on the other hand, can yield information on all three flavors.\\

The SNO, standing for Sudbury Neutrino Observatory, was a neutrino observatory located 2100 m underground in INCO's Creighton Mine in Sudbury, Ontario, Canada. It is a 1000-ton heavy-water ($ \text{D}_2\text{O} $) tank with the baseline of detecting high energy $ ^8\text{B} $ neutrinos in all of their flavors, thanks to three different processes:
\begin{enumerate}
	\item CC interaction:
	\begin{equation}
		\nu_e + d \rightarrow e^- + p + p,
	\end{equation}
	\item NC interaction:
	\begin{equation}
		\nu_x + d \rightarrow \nu_x + p + n,
	\end{equation}
	\item Elastic scattering:
	\begin{equation}
		\nu_x + e^- \rightarrow \nu_x + e^-,
	\end{equation}
\end{enumerate}
in which, $ x $ stands for $e$, $ \mu $, $ \tau $.\\

% Adding reference for this paper: http://journals.aps.org/prc/pdf/10.1103/PhysRevC.88.025501
The NC interaction channel allows the partly observation of $ \nu_\mu $ and $ \nu_\tau $ fluxes. The flux measured in SNO \cite{PhysRevC.88.025501}
\begin{equation}
	\Phi^{NC}_{\nu_{e, \mu, \tau}} = (5.25 \pm 0.16(\text{stat.})^{+0.11}_{-0.13}(\text{syst.})) \times 10^6~\text{cm}^{-2}\text{s}^{-1}, \nonumber
\end{equation}
while the electron neutrino flux predicted by solar model is
\begin{equation}
	\Phi_{\nu_e} = (5.88 \pm 0.65) \times 10^6~\text{cm}^{-2}\text{s}^{-1},
\end{equation}
for the BPS09(GS) model \cite{1538-4357-705-2-L123}, and
\begin{equation}
	\Phi_{\nu_e} = (4.85 \pm 0.58) \times 10^6~\text{cm}^{-2}\text{s}^{-1},
\end{equation}
for the BPS09(AGSS09) model \cite{1538-4357-705-2-L123}. In short, the total of all neutrino flavors observed on Earth is consistent with the electron flux predicted in solar models. Furthermore, the ratio between the electron neutrino flux and the total flux, all measured on Earth, will give us the survival probability:
\begin{equation}
	\dfrac{	\Phi_{\nu_e} }{	\Phi_{\nu_{e, \mu, \tau}}} = P(\nu_e\rightarrow\nu_e) = 0.317 \pm 0.016 \pm 0.009.
\end{equation}
This is a direct and model-independent  confirmation for the hypothesis of oscillation between electron neutrinos, produced in the Sun, into muon and tau neutrinos along their way toward the Earth. The model-dependent mass square difference values were also quoted in the paper \cite{PhysRevC.88.025501}. Specifically, for the two-flavor fit
\begin{equation}
	\Delta m^2_{21} = (5.6^{+1.9}_{-1.4})\times 10^{-5}~\text{eV}^2,
\end{equation}
and for the three-flavor fit
\begin{equation}
\Delta m^2_{21} = (7.46^{+0.2}_{-0.19})\times 10^{-5}~\text{eV}^2,
\end{equation}
with the mixing angle $\sin\theta_{13}$ measured to be
\begin{equation}
	\sin^2\theta_{13} = (2.49^{+0.20}_{-0.32})\times 10^{-2}.
\end{equation}

\subsubsection{Atmospheric neutrino problem}
Indeed, we are living under a giant accelerator and exposed to all kinds of tertiary particles. The Earth is continuously stroked with cosmic rays. These are primarily high-energy protons, contaminated with a small amount of heavy nuclei. Interacted with the atomic nuclei in the atmosphere, these cosmic rays produce a shower of hadronic particles. Taking up a large proportion among these secondary particles are pions and kaons\footnote{Kaons, due to their masses, produce neutrinos at slightly higher energy than the ones created in muon decays.} who, in turn, decay into muons and muon neutrinos. Eventually, the produced muons will complete the chain by decaying into muon neutrinos, electron neutrinos and electrons.
\begin{equation}
	\label{eq:piKdecay}
	\pi^\pm/K^\pm \rightarrow \mu^\pm + \nu_\mu (\bar{\nu}_\mu),
\end{equation}
\begin{equation}
	\label{eq:mudecay}
	\mu^\pm \rightarrow e^\pm + \nu_e (\bar{\nu}_e) + \nu_\mu (\bar{\nu}_\mu).
\end{equation}
The muons with energy lower than 1~GeV have a high probability to decay before touching the Earth. The ratio between the flux of muon neutrinos and that of electron neutrinos from the atmosphere is then expected to be around 2. Careful theoretical calculations predict the number within an uncertainty of 10-20\%. The muons with higher energy, which emerge as decay products of kaons and other charm hadrons, have a tendency to decay at or below the Earth's surface, so the  flux ratio mentioned above tends to increase. \\

However, there existed some hints about the deviation off the 2:1 ratio. In the middle of the 1980s, several proton decay experiments, such as the IMB in the USA or Kamiokande in Japan, were brought into operation. These are water Cherenkov detectors, located deep down underground to reduce the cosmic ray backgrounds. However, due to the highly penetrative ability, the detectors were exposed to atmospheric neutrinos and the upward-going muons produced in the interactions between atmospheric neutrinos and the surrounding rock. The measurement of the atmospheric neutrinos flux, thus, became substantial for the detection of nucleon decays. It turned out that the observed flux is significantly lower than the expected one. This deficit was somewhat peculiar because it was not seen in other experiments, namely NUSEX and Frejus.\footnote{NUSEX and Frejus are two fine-grained iron calorimeter and tracking, also used for proton decay detection, in Italia and France respectively.} Noticeably, the IMB experiment only observed the deficit for fully contained events with energy lower than 1.5~GeV. The partially contained events with energy above 0.95~GeV do not exhibit any significant deviation from the prediction. This peculiarity was then called the atmospheric neutrino anomaly.\\

\subsubsection{The Super-Kamiokande experiment}


	